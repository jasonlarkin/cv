\documentclass{article}
\synctex=1
\usepackage{fullpage}
\usepackage{amsmath}
\usepackage{amssymb}
\usepackage[hidelinks,colorlinks=false]{hyperref}
\usepackage{helvet}
\usepackage[margin=0.75in]{geometry}
\usepackage{enumitem}
%\setitemize{noitemsep,topsep=0pt,parsep=0pt,partopsep=0pt}
\setitemize{leftmargin=10pt,itemindent=10pt, noitemsep, topsep=4pt, parsep=2pt, partopsep=2pt}
\renewcommand{\familydefault}{\sfdefault}
\textheight=10in
\pagestyle{empty}
%\raggedbottom
\raggedright

%  \renewcommand{\encodingdefault}{cg}
  %\renewcommand{\rmdefault}{lgrcmr}

\def\bull{\vrule height 0.8ex width .7ex depth -.1ex }
% DEFINITIONS FOR RESUME
\newcommand{\area}[2]{\vspace*{-9pt} \begin{verse}\textbf{#1}   #2 \end{verse}  }
\newcommand{\lineunder}{\vspace*{-8pt} \\ \hspace*{-18pt} \hrulefill \\}
\newcommand{\header}[1]{{\hspace*{-15pt}\vspace*{6pt} \textsc{#1}} \vspace*{-6pt} \lineunder}
\newcommand{\employer}[3]{{ \textbf{#1} (#2) \underline{\textbf{\emph{#3}}}\\  }}
\newcommand{\contact}[3]{
\vspace*{-8pt}
\begin{center}
{  {#1}}\\
#2 \lineunder 
#3
\end{center}
\vspace*{-8pt}
}
\newenvironment{achievements}{\begin{list}{$\bullet$}{\topsep 0pt \itemsep -2pt}}{\vspace*{4pt}\end{list}}
\newcommand{\schoolwithcourses}[3]{
 \textbf{#1} #2 $\bullet$ #3\\ 
\vspace*{5pt}
}
\newcommand{\school}[3]{
 \textbf{#1} #2 $\bullet$ #3\\
}
% END RESUME DEFINITIONS


%tikz
\usepackage{tikz}
\usetikzlibrary{shapes,arrows}

\begin{document}

\small
\smallskip
\vspace*{-44pt}

\contact{\href{http://jasonlarkin.org}{\LARGE{Jason M Larkin, MS, P}\lowercase{h}\LARGE{D}, Consultant, Founder}}
{\href{mailto:jasonlarkin84@gmail.com}{jasonlarkin84@gmail.com} $\bullet$ 
\href{http://jasonlarkin.github.io/}{jasonlarkin.github.io} $\bullet$
\href{http://phdmentors.org/}{phdmentors.org}}

%$\bullet$ 
%\textbf{Note: }\href{https://github.com/jasonlarkin/cmu-doc/raw/master/cv/stash/jasonlarkin_cv.pdf}
%{Red are hyperlinks in digital copy}}

\header{\large{Career Overview}}

\textbf{I learn quickly and seek complex problems}. 
\begin{itemize}
\item I have extensive experience in \textbf{\href{http://jasonlarkin.github.io/}{research and development}} in diverse  fields, collaborating in multi-disciplinary teams globally, delivering my research in \textbf{\href{http://jasonlarkin.org/pub.html}{publications} and \href{http://jasonlarkin.org/pres.html}{presentations}}. 
\item Simultaneously, I am transferring this research and development into \textbf{\href{https://www.phdmentors.org/}{startup companies}}.  

\item I have done the following "deep dives":
\begin{itemize}
\item \textbf{Quantum Computing and Information Science}
\item \textbf{High Performance Cloud Computing and Collaboration}:CMU-SEI QHub,PhdMentors.org,complexityweekend.com
\item \textbf{Materials Science}:atomistic$/$molecular modeling,nanoscale transport 
\item \textbf{Condensed Matter Physics}:turbulence$/$microfluidics,nonlinear$/$visco-elasticity
\item \textbf{Knowledge Management Systems}:natural language processing/understanding,"umwelt hacking"
%\item CyberPhysical:robotics$/$autonomy$/$assurance,"umwelt hacking"
\end{itemize}

\end{itemize}

%\href{http://jasonlarkin.org}{\normalsize{\textbf{I learn quickly and seek complex problems}}}. I specialize in multi-\href{https://en.wikipedia.org/wiki/Multiscale_modeling}{scale} / \href{https://en.wikipedia.org/wiki/Multiphysics}{physics} modeling and prediction with varying levels of complexity (i.e., \href{https://en.wikipedia.org/wiki/Back-of-the-envelope_calculation}{"back of the envelope"} versus \href{https://en.wikipedia.org/wiki/Supercomputer}{computationally-intensive} \href{http://jasonlarkin.github.io/projects-phd-mfp.html}{simulation}). I have extensive experience performing \href{http://jasonlarkin.org/research}{research} and \href{http://jasonlarkin.org/development}{development} in \href{http://jasonlarkin.org/diverse_fields}{diverse fields} and have   collaborated in \href{http://jasonlarkin.org/multi-disciplinary}{multi-disciplinary} teams \href{http://jasonlarkin.org/globe}{across the globe}. I have delivered the results of my research  through \href{http://jasonlarkin.org/pub.html}{publication} and \href{http://jasonlarkin.org/pres.html}{public speaking}. \textbf{I am seeking a position to utilize and increase my knowledge of complex modeling and research.}

%The software development has covered the \href{https://www.google.com/search?q=full+stack+software&oq=full+stack+software&aqs=chrome..69i57.2623j0j7&sourceid=chrome&es_sm=122&ie=UTF-8}{"full-stack"}, was produced using \href{https://en.wikipedia.org/wiki/Agile_software_development}{Agile} methods, and has resulted in \href{http://jasonlarkin.org/research_scalable}{scalable and sustainable research}.

%https://www.youtube.com/watch?v=rjbEWeu2Nwc&feature=youtu.be#t=51m53s

%My expertise includes complex systems modeling,%
%and would like to continue...studies of complex systems...such as economic, biological,...
%and open-source collaboration to improve the way research is performed and results are disseminated.

\header{\large{Experience}}


\employer{\href{https://www.sei.cmu.edu/about/divisions/emerging-technology-center/}{CMU Software Engineering Institute Emerging Technology Center}}{2017 - Present}{Senior Research Scientist}

%(2013-2015)\underline{\textbf{\emph{Software Engineer}}}

\vspace{1mm}
    \begin{itemize}
      \item \textbf{PROJECTS}
      \begin{itemize}
        \item \href{https://www.sei.cmu.edu/our-work/quantum-computing/index.cfm}{\textbf{Quantum Advantage Evaluation Framework}}
        \begin{itemize}
          \item PI (1.6M funding) for research in applications of quantum advantage versus classical state-of-the-art computing
          \item Applications in combinatorial optimization, materials science, machine learning, cryptography
          \item Created \href{https://www.sei.cmu.edu/our-work/quantum-computing/index.cfm}{CMU-SEI QHub}, supporting 10 researchers, 5 research publications         
        \end{itemize}
        \item \textbf{\href{https://www.darpa.mil/program/software-defined-hardware}{Software Defined Hardware (SDH, \href{http://www.darpa.mil/default.aspx}{DARPA})}}
        \begin{itemize}
            \item Worked to create the testing infrastructure for SEI via AWS, JupyterHub, and assorted compilation and analysis tools (Intel, PyTorch, Tensorflow, ARM) to establish maximum theoretical and empirical performance for data-intensive workflows (machine learning, graph analytics, optimization) on commodity hardware (CPU, GPU, TPU). 
        \end{itemize}
        
        \item \textbf{GraphBLAS Test Framework}
        \begin{itemize}
            \item GraphBLAS Test Framework (Scott McMillan PI): 
Created a test framework for multiple implementations of GraphBLAS.org, including \href{https://github.com/cmu-sei/gbtl}{SEI’s GraphBLAS Template Library}

        \end{itemize}
        
        \end{itemize}
    \end{itemize}

\employer{PhDMentors}{2019 - present}{Co-Founder}

    \begin{itemize}

      \item \textbf{PROJECTS}
        \begin{itemize}
          \item Fully cloud$/$virtual research mentoring service, O(10) mentors, O(40) mentees with clear efficacy$/$market fit.  
          \item Formulating new business model, partnership with complexityweekend.com
        \end{itemize}
    \end{itemize}

\employer{\href{http://spiralgen.com/}{SpiralGen, Inc.}}{2013 - 2017}{Senior Research Engineer}

%(2013-2015)\underline{\textbf{\emph{Software Engineer}}}

\vspace{1mm}
%\par
%My work involved research and development across the of software and hardware that supported the Spiral toolchain for , with a focus on delivering  using (see \textbf{Skills} section for more):

%My focus of expertise is identified in \textbf{bold}.
%    \item Spanned range of languages from "nearly-assembly" C to JavaScript/Python/etc.  
    \begin{itemize}

      \item \textbf{PROJECTS}

      Supported work of commercial and research projects featuring the code-generation engine \textbf{\href{http://spiral.net/}{Spiral}}.

      \begin{itemize}

        \item \textbf{Spiral Code Generation Toolbox for Matlab/Simulink and Advanced Driver Assistance Systems (ADAS)}:Developed toolbox for Spiral code generation of Automotive Adaptive Cruise Control Using FMCW and MFSK Technology.

        \item \textbf{\href{http://www.darpa.mil/Our_Work/I2O/Programs/High-Assurance_Cyber_Military_Systems_(HACMS).aspx}{High-Assurance Cyber Military Systems (HACMS, \href{http://www.darpa.mil/default.aspx}{DARPA})}}: Automatically-optimized / \href{https://en.wikipedia.org/wiki/Formal_methods}{formally-verified} kernels for \textbf{\href{https://en.wikipedia.org/wiki/Cyber-physical_system}{Cyber-Physical Systems}} using Spiral, plug-in for OSATE and the \textbf{Architecture Analysis $\&$ Design Language (AADL)}, \textbf{DARPA Demo Days} ground/air vehicles, virtual/physical environments, \textbf{Large/diverse} collaboration team interacting with O(1000K) Lines of Code (LOC). 

      \item \textbf{SpiralFFT for \href{http://www.ncsa.illinois.edu}{National Center for Supercomputing  (NCSA)} \href{http://www.ncsa.illinois.edu/enabling/bluewaters}{Blue Waters}}: Improve petascale performance of \textbf{Hybrid \href{http://en.wikipedia.org/wiki/Message_Passing_Interface}{MPI} / \href{http://en.wikipedia.org/wiki/OpenMP}{OpenMP}} \href{https://en.wikipedia.org/wiki/Fast_Fourier_transform}{FFT} and Stencils using Spiral. \textbf{\href{https://en.wikipedia.org/wiki/Pseudo-spectral_method}{Pseudo Spectral Methods}} for modeling turbulence and the \textbf{NEURON} simulation environment. 

    \item \textbf{SpiralGen DevOps and Cloud Infrastructure}: \textbf{\href{https://www.google.com/search?q=full+stack+software&oq=full+stack+software&aqs=chrome..69i57.2623j0j7&sourceid=chrome&es_sm=122&ie=UTF-8}{"Full-stack"}} software development for \href{https://en.wikipedia.org/wiki/Supercomputer}{high-performance}, \href{https://en.wikipedia.org/wiki/Embedded_system}{embedded}, and \href{https://en.wikipedia.org/wiki/Cloud_computing}{cloud computing}). \textbf{\href{https://en.wikipedia.org/wiki/Agile_software_development}{Agile}} solutions in a \textbf{\href{http://en.wikipedia.org/wiki/Continuous_integration}{Continuous Integration}} using \textbf{\href{https://en.wikipedia.org/wiki/Software_configuration_management}{Software Configuration Management}} (SCM). \textbf{\href{https://dzone.com/articles/who-needs-online-ide}{WebIDE}} interface using \textbf{\href{https://en.wikipedia.org/wiki/Virtual_machine}{Virtual Machines}} (VMs) containers on \textbf{\href{https://en.wikipedia.org/wiki/Amazon_Web_Services}{Amazon Web Services}} (AWS).  Integration: \href{http://www.mathworks.com/products/matlab/}{Matlab}/\href{http://www.mathworks.com/products/simulink/}{Simulink}/\href{http://www.mathworks.com/help/matlab/ref/mex.html}{Mex}, \href{https://www.python.org/}{Python}/\href{http://cython.org/}{Cython}, \href{http://www.ros.org/}{ROS}, \href{https://www.cyberbotics.com/overview}{Webots}, \href{http://www.ls.cs.cmu.edu/KeYmaeraX/}{KeyMaeraX}, \href{http://www.wolfram.com/mathematica/}{Mathematica}.


    \item \textbf{\href{http://www.darpa.mil/Our_Work/MTO/Programs/Power_Efficiency_Revolution_for_Embedded_Computing_Technologies_(PERFECT).aspx}{Power Efficiency Revolution for Embedded Computing Technologies (PERFECT, \href{http://www.darpa.mil/default.aspx}{DARPA})}}Eclipse RCP \textbf{first commerical release} of \href{http://www.spiralgen.com/new-products/spiral-fft-gpl-10}{SpiralFFT}.  

    \item \textbf{\href{http://www.darpa.mil/program/building-resource-adaptive-software-systems}{Building Resource Adaptive Sotware Systems} (BRASS, \href{http://www.darpa.mil/default.aspx}{DARPA})}:Test harness for Spiral-generated resource adaptive FFT for Synthetic Apeture Radar. 

  \end{itemize}

\item \textbf{FUNDING AND RESOURCE PROPOSALS}
        \begin{itemize}
          \item \textbf{DOD 172-008 SBIR} (co-wrote Phase 1).
          \item \textbf{\href{http://science.energy.gov/sbir/}{DOE SG-13808 SBIR}} (co-wrote, \href{https://www.osti.gov/scitech/servlets/purl/1242462}{Phase 1} awarded, Phase 2 submitted, denied).
          \item \textbf{DOD A15-102 SBIR} (PI, Phase 1 submitted).
          \item \textbf{\href{https://bluewaters.ncsa.illinois.edu/paid-ime-submissions}{NSF NCSA Blue Waters PAID IME Submission}} (Co-PI).
          \item \textbf{Optimization of 3-D FFTs for Intel Xeon Phi and NVIDIA Kepler K20 GPUs using Spiral} (PI, awarded).
        \end{itemize}


    \end{itemize}

%\end{achievements}


\employer{\href{https://www.cmu.edu/me/}{Carnegie Mellon University}}{2010-2012}{TA-Heat Transfer: lectured on conduction, convection, radiation.}
%	\begin{achievements}
%  \begin{itemize}
%    \item 
%  \end{itemize}
%	\end{achievements}

\employer{\href{http://www.engineering.pitt.edu/MEMS/}{University of Pittsburgh}}{2008}{TA-Fluid Mechanics: viscous, boundary, scale similarity, dimensional analysis.}
%  \begin{achievements}
%  \begin{itemize}
%    \item 
%  \end{itemize}
%  \end{achievements}

\employer{\href{http://www.physicsandastronomy.pitt.edu/}{University of Pittsburgh}}{2007-2009}{Lecturer-Physics: mathematics, turbulence, statistics and nonlinearity.}
%  \begin{achievements}
%  \begin{itemize}
%    \item Lectured on mathematics, bio-physics, turbulence, statistical and nonlinear phenomena. 
%  \end{itemize}
%  \end{achievements}

\employer{\href{http://www.precisiontherapeutics.com/}{Precision Therapeutics}}{2006-2007}{Intern-Technology Development: optical microscope automation design.} 
%  \begin{achievements}
%  \begin{itemize}
%    \item  
%  \end{itemize}
%  \end{achievements}

\header{\large{Education}}

\begin{itemize}[leftmargin=*]
  \item \textbf{Carnegie Mellon University} Pittsburgh, PA PhD Mechanical Engineering, 2013 GPA: 3.9/4.0
%\schoolwithcourses{Carnegie Mellon University}{Pittsburgh, PA}{PhD Mechanical Engineering, 2013 GPA: 3.85/4.00}
%\area{
  \begin{itemize}
    \item \textbf{Thesis}: \href{http://jasonlarkin.github.io/projects-phd.html}
    {Vibrational Mode Properties of Disordered Solids from High-Performance Atomistic Simulations.}
%Numerically investigated thermal properties of crystal alloys, glasses, and organic materials using classical and \href{http://en.wikipedia.org/wiki/Ab_initio_quantum_chemistry_methods}{\emph{ab initio}}-based atomistic techniques.
% Information Content: 1-10 GB
    \item \textbf{\href{http://ntpl.me.cmu.edu/research.html}{Nanostructure Thermal Conductivity}:} investigator for  
\href{http://www.wpafb.af.mil/afrl/afosr/}{AFOSR} on the \href{http://www.hpcmo.hpc.mil/cms2/index.php}{DOD's HPCMP}.
    \item \textbf{\href{https://nanochemistry.curtin.edu.au/local/docs/gulp/gulp4.2_manual.pdf}{GULP}:} \textbf{international} collaboration with \href{http://nanochemistry.curtin.edu.au/people/staff.cfm/J.Gale}{Julian Gale} at the 
\href{http://nanochemistry.curtin.edu.au/}{Nanochemistry Research Institute} at \href{http://www.curtin.edu.au/}{Curtin University}.
  \end{itemize}

  \item \textbf{University of Pittsburgh} Pittsburgh, PA MS Mechanical Engineering, 2009 GPA: 3.7/4.0
  \begin{itemize}
    \item \textbf{Thesis}: \href{http://jasonlarkin.github.io/projects-ms.html}{Statistics of Particle Concentrations in Free-Surface Turbulence.} 
    \item \textbf{\href{http://jasonlarkin.github.io/projects-ms.html}{Statistics of Free-Surface Turbulence}:} \textbf{international} collaboration with \href{http://perso.ens-lyon.fr/alain.pumir/Pumir_webpage.html}{Alain Pumir} and \href{https://groups.oist.jp/ciu/mahesh-m-bandi}{Mahesh M. Bandi}. 
%Performed experiments using novel 2D and 3D flow configurations to study turbulence as a nonlinear dynamical system. 
  \end{itemize}

  \item \textbf{University of Pittsburgh} Pittsburgh, PA BS Mechanical Engineering, 2007 GPA: 3.2/4.0
  \begin{itemize}
    \item \textbf{Research}: \href{http://en.wikipedia.org/wiki/Finite_element_method}{FEM} design of model arterial bifurcation.
  \end{itemize}

  \item \textbf{Steel Center AVTS} Jefferson Hills, PA \href{http://en.wikipedia.org/wiki/Computer-aided_design}{CADD} Certification, 2002 GPA: 3.80/4.00

\end{itemize}

\header{\Large{Skills/Tools}}

%\begin{achievements}

\begin{itemize}[leftmargin=*]

  \item \textbf{Publication and Public Speaking}: \href{https://scholar.google.com/citations?hl=en&user=ry2klw0AAAAJ}{google scholar} (journal pubs (18), book chapters (2), conference presentations (28).

  \item \textbf{"Full-Stack" Software Engineering (stacks)}: 
  \begin{itemize}
    \item Python/C++/C (PyTorch/Tensforflow/NLTK/scipy/numpy)
    \item Matlab-Simulink/C++/C/Fortran    
    \item \textbf{Software Configuration Management}: svn, git, GitHub, Jenkins, JIRA.  \textbf{Compilers/Compilation}: GNU, Intel, Visual Studio, MinGW, Cray, Cython, Mex, Ant, make, cmake, catkin$\_$make, MSBuild, Maven. 
    \item \textbf{Cloud Computing}: Amazon Web Service (AWS), Docker, VirtualBox/VMWare, Ubuntu, Red Hat, CentOS, CoreOS, Windows (XP, 7, 8, Server). MPI / OpenMP, Vector Intrinsics (SSE/AVX/etc), CoArray Fortran
  \end{itemize}

  \item \textbf{Hardware:} optics/lasers, DI/DO AI/AO interfaces, automation, machining, circuitry,  robotics control.
 
\end{itemize}


\end{itemize}
%\end{achievements}

\header{\Large{\href{http://jasonlarkin.github.io/pub.html}{Publications} (selected, 27 total)}}
%\begin{achievements}
\begin{itemize}[leftmargin=*]
\item Evaluation of Quantum Approximation Optimization Algorithm, 
J Larkin, et al, arXiv preprint arXiv:2006.04831 (2020)
\item Achieving the Quantum Advantage in Software, \href{https://insights.sei.cmu.edu/sei_blog/2019/12/achieving-the-quantum-advantage-in-software.html}{SEI Blog, (2019)} 
\item Reduced thermal conductivity of Si/Ge random layer nanowires,
N Samaraweera, JM Larkin, Journal of App. Physics (2018)
\item Thermal conductivity accumulation in amorphous silica and silicon
JM Larkin, et al, Physical Review B 89 (14), 144303 (2014)
%\item Decorrelating a Compressible Turbulent Flow: an Experiment, \href{http://pre.aps.org/abstract/PRE/v82/i1/e016301}{Physical Review E 82, 016301 (2010).}
\item Power-law distributions of particle concentration in free-surface flows
J Larkin, et al, Physical Review E 80 (6), 066301 (2009)
\end{itemize}
%\end{achievements}

\header{\Large{\href{http://jasonlarkin.org/pres.html}{Presentations} (selected, 28 total)}}
%\begin{achievements}
\begin{itemize}[leftmargin=*]
\item Evaluation of QAOA, J. Larkin(speaker), \href{https://www.youtube.com/watch?v=I7yp-qXk0Zo}{Association Quantum}, \href{https://www.meetup.com/Washington-Quantum-Computing-Meetup/events/272829277/}{DC Quantum Meetup 2020}. 
\item Quantum Circuit Optimization with SPIRAL, S. Mionis, J. Larkin, et al, Supercomputing 2020 (best presentation noimnee). 
\item Assesing Objective Functions for Quantum Variational Optimization, M. Jonsson, J. Larkin et al, IEEE Quantum Week 2020. 
\item Projecting NISQ-era quantum advantage with QAOA
GG Guerreschi, J Larkin, et al American Physical Society 65 2020.
\item \href{https://bluewaters.ncsa.illinois.edu/documents/10157/5a0a0d37-95bf-460b-a7f0-cfadd15abec8}{SpiralFFT for Blue Waters}, \textbf{J.M. Larkin} \href{https://www.youtube.com/watch?v=rjbEWeu2Nwc&feature=youtu.be#t=51m53s}{\textbf{(speaker)}}, \href{https://bluewaters.ncsa.illinois.edu/paid-ime#SPIRAL FFT}{NCSA Symposium for Petascale 2015}.
\item Virtual Crystal Approximation, \textbf{J.M. Larkin (speaker)}, \href{http://www.mrs.org/spring2013/}{2013 MRS Spring Meeting} San Francisco, CA.
\item Generalized Fractal Dimensions...Turbulence, \textbf{J.M. Larkin (speaker)}, \href{http://meetings.aps.org/Meeting/MAR08/Content/1017}{2008 American Physical Society March Meeting}.
\item Flow Chamber to Explore the Development of Cerebral Aneurysms, J. Larkin, et al, 2007 Biomedical Engineering Society. 
%PHONONS 2012 http://www-personal.umich.edu/~pipe/Phonons_2012_Abstract_Book.pdf
\end{itemize}
%\end{achievements}


%\header{\Large{Honors}}
%%\begin{achievements}
%\begin{itemize}[leftmargin=*]
%\item \href{http://www.asmeconferences.org/HT2012/}{2012 ASME MHNMT International Summer Heat Transfer Conference} \textbf{Top 5 Technical Paper}
%\item \href{http://www.cmu.edu/me/news/archive/2011/bennett-conference.html}{2011 Bennett Conference \textbf{Best Presentation}}
%\item \href{http://www.ices.cmu.edu/newsitem.asp?NewsID=749}{2011 ICES \textbf{Northrop-Gruman Fellow}}
%\item 2007-2009 \textbf{NSF Graduate Student Research Grant} University of Pittsburgh Department of Physics.
%\end{itemize}
%\end{achievements}

%\header{Memberships}
%%\begin{achievements}
%\begin{itemize}[leftmargin=*]
%\item American Physical Society $\cdot$ American Society of Mechanical Engineers 
%$\cdot$ Materials Research Society $\cdot$ Society of Industrial and Applied Mathematics $\cdot$ DOD High Performance Computing Modernization Program $\cdot$ NCSA Blue Waters PAID IME
%\end{itemize}
%%\end{achievements}

\end{document}
