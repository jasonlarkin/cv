\documentclass{article}
\usepackage{fullpage}
\usepackage{amsmath}
\usepackage{amssymb}
\usepackage[dvipdfmx,hidelinks,colorlinks=true]{hyperref}
\usepackage{helvet}
\usepackage[margin=0.75in]{geometry}
\renewcommand{\familydefault}{\sfdefault}
\textheight=10in
\pagestyle{empty}
%\raggedbottom
\raggedright

%  \renewcommand{\encodingdefault}{cg}
  %\renewcommand{\rmdefault}{lgrcmr}

\def\bull{\vrule height 0.8ex width .7ex depth -.1ex }
% DEFINITIONS FOR RESUME
\newcommand{\area}[2]{\vspace*{-9pt} \begin{verse}\textbf{#1}   #2 \end{verse}  }
\newcommand{\lineunder}{\vspace*{-8pt} \\ \hspace*{-18pt} \hrulefill \\}
\newcommand{\header}[1]{{\hspace*{-15pt}\vspace*{6pt} \textsc{#1}} \vspace*{-6pt} \lineunder}
\newcommand{\employer}[3]{{ \textbf{#1} (#2) \underline{\textbf{\emph{#3}}}\\  }}
\newcommand{\contact}[3]{
\vspace*{-8pt}
\begin{center}
{\LARGE \scshape {#1}}\\
#2 \lineunder 
#3
\end{center}
\vspace*{-8pt}
}
\newenvironment{achievements}{\begin{list}{$\bullet$}{\topsep 0pt \itemsep -2pt}}{\vspace*{4pt}\end{list}}
\newcommand{\schoolwithcourses}[3]{
 \textbf{#1} #2 $\bullet$ #3\\ 
\vspace*{5pt}
}
\newcommand{\school}[3]{
 \textbf{#1} #2 $\bullet$ #3\\
}
% END RESUME DEFINITIONS


%tikz
\usepackage{tikz}
\usetikzlibrary{shapes,arrows}

\begin{document}

\small
\smallskip
\vspace*{-44pt}

\contact{\href{http://jasonlarkin.org}{Jason M Larkin}}
{\href{mailto:jasonlarkin84@gmail.com}{jasonlarkin84@gmail.com} $\bullet$ 
\href{http://jasonlarkin.org}{http://jasonlarkin.org} $\bullet$ 
\textbf{Note: }\href{https://github.com/jasonlarkin/cmu-doc/raw/master/cv/stash/jasonlarkin_cv.pdf}
{Red are hyperlinks in digital copy}}

\header{Career Overview}

\textbf{I learns things quickly}. I ...specialize in multi-scale, multi-physics,...) modeling and prediction (i.e., \href{https://en.wikipedia.org/wiki/Back-of-the-envelope_calculation}{"back of the envelope"} estimates versus \href{http://jasonlarkin.org/}{computationally-intensive simulation}).

I have extensive exepereience performing \href{http://jasonlarkin.org/research}{research} and \href{http://jasonlarkin.org/development}{development} in \href{http://jasonlarkin.org/diverse_fields}{diverse fields}, colloborating in \href{http://jasonlarkin.org/collaboration}{large} \href{http://jasonlarkin.org/multi-disciplinary}{multi-disciplinary} teams \href{http://jasonlarkin.org/globe}{across the globe}...\href{http://jasonlarkin.org/research_scalable}{scalable} and \href{http://jasonlarkin.org/research_sustainable}{sustainable} software engineering...\href{http://jasonlarkin.org/research}{public speaking}...\href{}{publication}...

\href{http://jasonlarkin.org}{I seek complex problems}. 

\par

\header{Skills}
\begin{achievements}

I am proficient in (descending order)...

\item \textbf{Engineering and Condensed Matter Physics}:(quantum physics (chemistry, statistical, field), solid-state physics, statistical physics (thermal, fluid,  ), thermal physics,  , fluid 
dynamics (turbulence,  microfluidics, biological), solid mechanics)

\item \textbf{Complex Modeling and Computation}:( Condensed Matter Physics , chaos and nonlinear phenomena,  robotics, genetics)

\item \textbf{Computation and "Full-Stack" Software Engineering}:(Matlab(20000), Python (10000),  HPC,  )

\item \textbf{Research and Development} (Journal Publication (10), Conference Presentation (20), Invited Presentation (10-100), Book 
Publication (2) )

\par
\item \textbf{Languages (Lines of Code): } Matlab (20000), Python (10000), Perl (1000), JavaScript (4500), Java (1000), C++/C (4500), Shell (?), Fortran (1000). \textbf{Misc}: \LaTeX, Markdown, HTML, XML, JSON, CSS

\item \textbf{Development:} \href{http://en.wikipedia.org/wiki/Software_configuration_management}{SCM} (\href{http://subversion.apache.org/}{svn}, \href{http://git-scm.com/}{git}, \href{http://jenkins-ci.org/}{Jenkins}). \textbf{Compilers/Compilation}: \href{http://gcc.gnu.org/}{GNU}, \href{https://software.intel.com/en-us/c-compilers}{Intel C/C++}, Visual Studio, \href{http://www.mingw.org/}{MinGW}, \href{http://cython.org/}{Cython}, \href{http://www.mathworks.com/help/matlab/ref/mex.html}{Mex}, \href{http://ant.apache.org/}{Ant}.\href{http://www.gnu.org/software/make/}{make},\href{http://www.cmake.org/}{cmake}, \href{http://wiki.ros.org/catkin/commands/catkin$\_$make}{catkin$\_$make},\href{http://en.wikipedia.org/wiki/MSBuild}{MSBuild},\href{http://maven.apache.org/}{Maven}, \href{http://www.mathworks.com/help/matlab/ref/mex.html}{Mex}, \href{http://cython.org/}{Cython}. \textbf{Integrated Environments}: \href{http://www.visualstudio.com/}{Visual Studio}, \href{http://www.eclipse.org/}{Eclipse}, \href{http://www.eclipse.org/}{Matlab/Simulink}, ROS (roscore, rospy, etc.). \textbf{Documentation}: \href{http://www.doxygen.org}{Doxygen}, docco, lex/flex

\item \textbf{High-Performance Computing:} \href{http://en.wikipedia.org/wiki/Beowulf_cluster}{Linux/Unix cluster administration/computing}, parallel computation (\href{http://en.wikipedia.org/wiki/Message_Passing_Interface}{MPI}, \href{http://en.wikipedia.org/wiki/OpenMP}{OpenMP}), \href{http://en.wikipedia.org/wiki/Streaming_SIMD_Extensions}{SSE}/\href{http://en.wikipedia.org/wiki/Advanced_Vector_Extensions}{AVX} vectorization.

\item \textbf{Deployment:} \href{http://aws.amazon.com/}{Amazon Web Services} EC2, Virtualization (\href{https://www.virtualbox.org/}{VirtualBox}, \href{http://www.vmware.com/}{VMWare}), Debian, \href{http://nsis.sourceforge.net/Main_Page}{NSIS} 

\item \textbf{Operating Systems}: Linux/Unix: (Ubuntu, Red Hat, CentOS, Mac), Windows XP,7,8,Server

\item \textbf{General Computing:}  Microsoft Office, Libre/Open Office, \href{http://www.gimp.org/}{GIMP}.

\item \textbf{Open-Source Development:} \href{http://github.com/jasonlarkin}{Github}, \href{http://projects.ivec.org/gulp/}{GULP}, \href{http://lammps.sandia.gov/}{LAMMPS}, \href{http://www.ros.org/}{ROS}, \href{http://arxiv.org/find/physics/1/au:+Larkin_J/0/1/0/all/0/1}{arXiv}.

\item \textbf{Modeling:} atomistic simulation, quantum chemistry, nanoscale transport, statistical and nonlinear analysis, computational fluid dynamics. 

\item \textbf{Hardware:} optics/lasers, DI/DO AI/AO interfaces, simple automation, machining, circuitry, simple robotics control.
\end{achievements}





%My expertise includes complex systems modeling,%
%and would like to continue...studies of complex systems...such as economic, biological,...
%and open-source collaboration to improve the way research is performed and results are disseminated.

\header{Experience}

\employer{\href{http://spiralgen.com/}{SpiralGen, Inc.}}{2015- , 2013-2015}{Senior Research Engineer , Software Engineer}

%(2013-2015)\underline{\textbf{\emph{Software Engineer}}}

\vspace{1mm}
\par
\textbf{\href{http://spiral.net/}{Spiral}: toolchain that creates automatically-optimized/platform-tuned and formally-verified  numerical kernels for Cyber-Physical Systems.}
\vspace{1mm}
My work at involved Research and Development across the \href{https://www.google.com/search?q=full+stack+software&oq=full+stack+software&aqs=chrome..69i57.2623j0j7&sourceid=chrome&es_sm=122&ie=UTF-8}{"full-stack"} of software and hardware that supported the Spiral toolchain over a range of projects. 


My study of complex systems includes, and has been assisted by, the "full-stack" of computing software and hardware for applications in high-performance, embedded, and cloud-based computing, with a focus on devlivering \href{https://en.wikipedia.org/wiki/Agile_software_development}{Agile} solutions. I have extensive experience delivering the results of these studies through publication (averaging nearly 2 (blah) journal articles per year) and \href{https://www.youtube.com/watch?v=rjbEWeu2Nwc&feature=youtu.be#t=51m53s}{public speaking}...

%My focus of expertise is identified in \textbf{bold}.

  \begin{itemize}
%    \item Spanned range of langauges from "nearly-assembly" C to JavaScript/Python/etc.  
    \item Software stack built and served to provide \href{http://en.wikipedia.org/wiki/Continuous_integration}{Continuous Integration  (CI)} of Spiral with its support tools using:
    \begin{itemize}
      \item Software Control Management (SCM): (\href{https://git-scm.com/}{git},svn,\href{https://github.com/spiralgen}{Github},\href{http://beanstalkapp.com/}{Beanstalk},\href{https://bitbucket.org/}{BitBucket},JIRA,\href{http://www.doxygen.org/}{Doxygen},JSdoc) 
      \item Virtual Machines (VMs) on Amazon Web Services (AWS) Elastic Compute Cloud (EC2): (AWS,EC2,\href{http://jenkins-ci.org/}{Jenkins},make/cmake/catkin$\_$make,Maven,)  performing code compilation/packaging
      \item Web-based Integrated Development Environment (WebIDE): (AWS,EC2,nginx,Docker,nodejs,mongodb,Mathematica,Matlab/Simulink,ROS,Webots,)
    \end{itemize}
  \end{itemize}

  Virtualized WebIDE provided environments to support the following \textbf{Projects}:

%\begin{achievements} 
        
\begin{itemize}

\item \textbf{\href{http://www.darpa.mil/Our_Work/I2O/Programs/High-Assurance_Cyber_Military_Systems_(HACMS).aspx}{High-Assurance Cyber Military Systems (HACMS, \href{http://www.darpa.mil/default.aspx}{DARPA})}}: Creates technology for the construction of high-assurance cyber-physical systems. 

  \begin{itemize}

    \item Developed, integrated and tested \href{http://spiral.net/}{Spiral}-generated \href{https://wiki.hh.se/wg211/images/e/e0/M13Franchetti.pdf}{HCOL} kernels with robot controller and monitors (\href{http://en.wikipedia.org/wiki/PID_controller}{PID}, 
\href{http://en.wikipedia.org/wiki/Euler_method}{Euler}) for Path Planning (SLAM, (Extended) Kalman filter), Navigation Control, PID controller 

    \item Simulated/Vritualized environments: \href{http://www.cyberbotics.com/}{Webots}(Docker,KeyMaeraX,Spiral,Mathematica,\href{https://www.acsac.org/2014/workshops/law/Shankar_LAW2014.pdf}{RADL})     \item Developed an .      \item Developed 
    \begin{itemize}
      \item 
      \item automated and virtualized (\href{http://www.ros.org/}{ROS}) test system for the Spiral-generated robot controller kernels and others ()
      \item created \href{http://www.darpa.mil/about-us/offices/i2o}{I2O}\href{http://fedscoop.com/darpa-showcases-projects-pentagon-courtyard}{Demo} (\href{http://www.ros.org/}{ROS}, \href{http://www.cyberbotics.com/}{Webots}, and \href{https://wiki.python.org/moin/PyQt}{PyQT}).
    \end{itemize}

    \item Physical targets: \href{http://blackirobotics.com/DARPA-SN-12-26_HACMS.php}{Black-i} \href{http://www.blackirobotics.com/LandShark_UGV_UC0M.html}{Landshark}, American Built Automobile, \href{http://smaccmpilot.org/}{SMACCM Quadcopter}. 

    \item Large/diverse collaboration team (\href{http://www.hrl.com/}{HRL}, \href{http://www.sri.com/}{SRI}, \href{http://www.cmu.edu/}{CMU}, \href{http://web.mit.edu/}{MIT}, \href{http://www.princeton.edu/}{Princeton}, \href{http://illinois.edu/}{UIUC}, \href{http://www.upenn.edu/}{UPenn}) using SCM (i.e., git) with complex integration of hardware and software (many branches, merges, etc.

  \end{itemize}

\item \textbf{\href{http://science.energy.gov/sbir/}{Department of Energy (DOE) Small Business Innovation Research (SBIR)}}

  \begin{itemize}
    \item Co-wrote SBIR Grant proposal. 
    \item Provided consultation on thermal and nuclear physics for...
    \item (Docker,KeyMaeraX,Spiral,Matlab/Simulink) Created  native (C/C++/Fortran) and interpreted (\href{http://cython.org/}{Cython} and \href{http://www.mathworks.com/help/matlab/ref/mex.html}{Mex}) example implementations of \href{http://spiral.net/}{Spiral}-generated code. 
  \end{itemize}

\item \textbf{\href{http://www.ncsa.illinois.edu}{National Center for Supercomputing A (NCSA)} \href{http://www.ncsa.illinois.edu/enabling/bluewaters}{Blue Waters}}:

  \begin{itemize}
    \item Improve...Petascale 
    \item Hybrid MPI OpenMP,  

enhancement and optimization of Spiral FFT for the \href{http://www.ncsa.illinois.edu/enabling/bluewaters}{Blue Waters Petascale supercomputer}

    \item Engagement and consulting with the science and engineering teams (including )

 The activity will also involve the enhancement of Spiral FFT code generation to meet the needs of science and engineering teams. SpiralGen, Inc. will perform the main development while CMU will provide scientific consulting and interaction with the application team

  \end{itemize}
  

\item \textbf{\href{http://www.darpa.mil/Our_Work/MTO/Programs/Power_Efficiency_Revolution_for_Embedded_Computing_Technologies_(PERFECT).aspx}{Power Efficiency Revolution for Embedded Computing Technologies (PERFECT, \href{http://www.darpa.mil/default.aspx}{DARPA})}}:(Docker,Spiral, VHDL, Verilog,)

  \begin{itemize}
    \item Virtualized Environment (VMWare, \href{http://cobbler.github.io/}{Cobbler}, \href{https://puppetlabs.com/}{Puppet}, \href{http://www.katello.org/}{Katello}) connected to physical targets (Intel Xeon Phi, ARM, etc.)
    \item Provide...
  \end{itemize}

\item Misc Projects
  \begin{itemize}
    \item Front-end evolved from \href{https://wiki.eclipse.org/index.php/Rich_Client_Platform}{Eclipse RCP} to cloud-based WebUI. 
    \item Windows Server 2008/2012 (Spiral,Matlab/Simulink,Python) \href{http://nsis.sourceforge.net/Main_Page}{NSIS}
  \end{itemize}

\end{itemize}

%\end{achievements}

\employer{Carnegie Mellon University}{2010-2012}{Teaching Assistant-Heat Transfer}
	\begin{achievements}
	\item Topics in conduction, convection, and radiation. Supervised recitations and substituted for lectures. 
	\end{achievements}

\employer{University of Pittsburgh}{2008}{Teaching Assistant-Advanced Fluid Mechanics}
	\begin{achievements}
	\item Topics in viscous flow, boundary layer theory, and scale similarity. 
	\end{achievements}

\employer{University of Pittsburgh}{2007-2009}{Lecturer-Physics}
	\begin{achievements}
	\item Lectured to students and faculty on mathematics, bio-physics, turbulence, statistical and nonlinear phenomena. 
	\end{achievements}

\employer{Precision Therapeutics}{2006-2007}{Intern-Technology Development}
	\begin{achievements}
	\item Worked with team of software developers and laboratory equipment specialists.
	\item Used \href{http://en.wikipedia.org/wiki/Computer-aided_design}{CAD} to design and fabricate components of optical microscopes and laboratory automation controls. 
	\end{achievements}

\header{Education}

\schoolwithcourses{Carnegie Mellon University}{Pittsburgh, PA}{PhD Mechanical Engineering, 2013 GPA: 3.85/4.00}

\area{
Thesis: \href{http://jasonlarkin.github.io/projects-phd.html}
{Vibrational Mode Properties of Disordered
Solids from High-Performance Atomistic
Simulations and Calculations.}}
{Numerically investigated thermal properties of crystal alloys, glasses, and organic materials using classical and \href{http://en.wikipedia.org/wiki/Ab_initio_quantum_chemistry_methods}{\emph{ab initio}}-based atomistic techniques. Information Content: 1 GB} 

\area{Coursework:}
{
statistical analysis $\cdot$ nonlinear optimization $\cdot$ numerical methods $\cdot$ molecular/electron structure $\cdot$ nanoscale transport phenomena  
}

\schoolwithcourses{University of Pittsburgh}{Pittsburgh, PA}{MS Mechanical Engineering, 2009 GPA: 3.70/4.00}

\area{
Thesis: \href{http://jasonlarkin.github.io/projects-ms.html}
{Statistics of Particle Concentrations in Free-Surface Turbulence.}}
{Performed experiments using novel 2D and 3D flow configurations to study turbulence 
as a nonlinear dynamical system.}

\area{Coursework:}
{
turbulence $\cdot$ chaos and nonlinear phenomena $\cdot$ complexity and information theory $\cdot$ quantum and statistical physics 
}

\schoolwithcourses{University of Pittsburgh}{Pittsburgh, PA}{BS Mechanical Engineering, 2007 GPA: 3.20/4.00}

\area{
Research: }
{Used \href{http://en.wikipedia.org/wiki/Finite_element_method}{FEM} to design a model arterial bifurcation for \emph{in vivo} study.}

\schoolwithcourses{Steel Center AVTS}{Jefferson Hills, PA}{\href{http://en.wikipedia.org/wiki/Computer-aided_design}{CADD} Certification, 2002 GPA: 3.80/4.00}
\area{
Coursework: }
{Trained in \href{http://en.wikipedia.org/wiki/Computer-aided_design}{CAD} using \href{http://www.autodesk.com/}{Autodesk}'s \href{http://www.autodesk.com/products/autocad/overview}{AutoCAD} (15.6) and \href{http://www.autodesk.com/products/autodesk-inventor-family/overview}{Inventor} (5.3) 
to produce machined products by \href{http://en.wikipedia.org/wiki/Computer-aided_manufacturing}{CAM} and human machining.  }




\header{Projects}
\begin{achievements} 

\item \href{http://blogs.ubc.ca/amerimech2014/files/2014/04/ameritech_mcgaughey_apr14.pdf}{Phonon Transport in Periodic Materials with Feature Sizes of 1 nm to 1 $\mu$m}

\item \textbf{\href{http://ntpl.me.cmu.edu/research.html}{Quantum Mechanics-Driven Prediction of Nanostructure Thermal Conductivity}:}
served as investigator under the 
\href{http://www.wpafb.af.mil/afrl/afosr/}{AFOSR} with collaborators at Carnegie Mellon and University of Pittsburgh, performing 
calculations on the \href{http://www.hpcmo.hpc.mil/cms2/index.php}{DOD's HPCMP}.

\item \textbf{\href{https://github.com/jasonlarkin/disorder}{disorder}:} a comprehensive repository of open-source code and data from my PhD thesis, hosted on \href{http://github.com/jasonlarkin}{Github}.

\item \textbf{\href{https://github.com/ntpl/ntpy}{ntpy}:} created this open-source  collaborative effort between members of \href{http://ntpl.me.cmu.edu/}{NTPL} and \href{http://www.mie.utoronto.ca/labs/atoms/}{University of Toronto}.

\item \textbf{\href{https://nanochemistry.curtin.edu.au/local/docs/gulp/gulp4.2_manual.pdf}{GULP}:} international collaboration with \href{http://nanochemistry.curtin.edu.au/people/staff.cfm/J.Gale}{Julian Gale} at the 
\href{http://nanochemistry.curtin.edu.au/}{Nanochemistry Research Institute} at \href{http://www.curtin.edu.au/}{Curtin University}.

\item \href{http://www.andrew.cmu.edu/user/caroling/publications.html}{Effective energy density and thermal diffusivity of Ioffe-Regel confined vibrations in amorphous silica}: collaboration with C. S. Gorham...

\item \href{Origins of thermal conductivity changes in strained crystals}{http://ntpl.me.cmu.edu/publications.html}: collaboration with \href{http://www.kdparrish.com/}{K. D. Parrish}

\item \href{https://tspace.library.utoronto.ca/bitstream/1807/42871/1/Huberman_Samuel_C_201311_MASc_thesis.pdf}{Phonon Properties in Superlattices}
\href{http://web.mit.edu/schuberm/www/}{Samuel Huberman} while at \href{http://www.mie.utoronto.ca/labs/atoms/}{University of Toronto}. (now at \href{http://web.mit.edu/nanoengineering/people/students.shtml}{NanoEngineering Group MIT}) supplied source code and expertise...to \href{http://jasonlarkin.org/pub.html}{publish}... 

\item Experimental Studies in 2D Turbulence
\href{http://www.physicsandastronomy.pitt.edu/news-story/phd-defense-stefanus}{S. Stefanus}

\item \textbf{\href{http://jasonlarkin.github.io/projects-ms.html}{Statistics of Free-Surface Turbulence}:} international collaboration with \href{http://perso.ens-lyon.fr/alain.pumir/Pumir_webpage.html}{Alain Pumir} at \href{http://www.ens-lyon.eu/annuaire/m-pumir-alain-83656.kjsp?RH=ZYZYZYZYZYZYZYZYZYZYZY}{ENS Lyon}, France and \href{https://groups.oist.jp/ciu/mahesh-m-bandi}{Mahesh M. Bandi} at 

\item \href{https://github.com/jasonlarkin/pcm-potentials}{Phase Change Materials MD potentials} using GULP. Nonlinear optimization...simulated annealing...genetic algorithm?..conjugate gradient...Newton/Rhapson... Density Functional Theory...fit to quantum mechanically derived energy hypersurfaces

\item pylitrev: uses Python Natural Language Toolkit (NLP) to provide...

\end{achievements}

\header{\href{http://jasonlarkin.github.io/pub.html}{Publications} (selected, 11 total)}
\begin{achievements}
\item "Origin of the Exceptionally Low Thermal Conductivity of Fullerene Derivative  PCBM Films", 
\href{http://jasonlarkin.github.io/projects-phd-pcbm.html}{(in progress).}
\item "Decorrelating a Compressible Turbulent Flow: an Experiment", \href{http://pre.aps.org/abstract/PRE/v82/i1/e016301}{Physical Review E 82, 016301 (2010).}
\end{achievements}

\header{\href{http://jasonlarkin.org/pres.html}{Presentations} (selected, 15 total)}
\begin{achievements}
\item \href{https://bluewaters.ncsa.illinois.edu/documents/10157/5a0a0d37-95bf-460b-a7f0-cfadd15abec8}{"SpiralFFT for Blue Waters"}, J. Larkin (speaker), T. Popovici, M. Franusich, F. Franchetti, \href{https://bluewaters.ncsa.illinois.edu/paid-ime#SPIRAL FFT}{NCSA Blue Waters Symposium for Petascale Science and Beyond} May 10-13, 2015
\item "Evaluation of the Virtual Crystal Approximation for Predicting Thermal Conductivity", J.M. Larkin (speaker), A.J.H.
   McGaughey, \href{http://www.mrs.org/spring2013/}{2013 MRS Spring Meeting} San Francisco, CA.
\item "The Generalized Fractal Dimensions of a 2-D Compressible Turbulence", J. Larkin (speaker), M. Bandi, W. Goldburg, \href{http://meetings.aps.org/Meeting/MAR08/Content/1017}{2008 American Physical Society March Meeting} New Orleans, LA.

%PHONONS 2012 http://www-personal.umich.edu/~pipe/Phonons_2012_Abstract_Book.pdf

\end{achievements}


\header{Honors}
\begin{achievements}
\item \href{http://www.asmeconferences.org/HT2012/}{2012 ASME MHNMT International Summer Heat Transfer Conference} Top 5 Technical Paper
\item \href{http://www.cmu.edu/me/news/archive/2011/bennett-conference.html}{2011 Bennett Conference Best Presentation}
\item \href{http://www.ices.cmu.edu/newsitem.asp?NewsID=749}{2011 ICES Northrop-Gruman Fellow}
\item 2007-2009 NSF Graduate Student Research Grant University of Pittsburgh Department of Physics.
\end{achievements}

\header{Memberships}
\begin{achievements}
\item American Physical Society $\cdot$ American Society of Mechanical Engineers 
$\cdot$ Materials Research Society $\cdot$ Society of Industrial and Applied Mathematics $\cdot$ DOD High Performance Computing Modernization Program
\end{achievements}

\end{document}
