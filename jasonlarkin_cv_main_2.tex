\documentclass{article}
\synctex=1
\usepackage{fullpage}
\usepackage{amsmath}
\usepackage{amssymb}
\usepackage[hidelinks,colorlinks=false]{hyperref}
\usepackage{helvet}
\usepackage[margin=0.75in]{geometry}
\usepackage{enumitem}
%\setitemize{noitemsep,topsep=0pt,parsep=0pt,partopsep=0pt}
\setitemize{leftmargin=10pt,itemindent=10pt, noitemsep, topsep=4pt, parsep=2pt, partopsep=2pt}
\renewcommand{\familydefault}{\sfdefault}
\textheight=10in
\pagestyle{empty}
%\raggedbottom
\raggedright

%  \renewcommand{\encodingdefault}{cg}
  %\renewcommand{\rmdefault}{lgrcmr}

\def\bull{\vrule height 0.8ex width .7ex depth -.1ex }
% DEFINITIONS FOR RESUME
\newcommand{\area}[2]{\vspace*{-9pt} \begin{verse}\textbf{#1}   #2 \end{verse}  }
\newcommand{\lineunder}{\vspace*{-8pt} \\ \hspace*{-18pt} \hrulefill \\}
\newcommand{\header}[1]{{\hspace*{-15pt}\vspace*{6pt} \textsc{#1}} \vspace*{-6pt} \lineunder}
\newcommand{\employer}[3]{{ \textbf{#1} (#2) \underline{\textbf{\emph{#3}}}\\  }}
\newcommand{\contact}[3]{
\vspace*{-8pt}
\begin{center}
{  {#1}}\\
#2 \lineunder 
#3
\end{center}
\vspace*{-8pt}
}
\newenvironment{achievements}{\begin{list}{$\bullet$}{\topsep 0pt \itemsep -2pt}}{\vspace*{4pt}\end{list}}
\newcommand{\schoolwithcourses}[3]{
 \textbf{#1} #2 $\bullet$ #3\\ 
\vspace*{5pt}
}
\newcommand{\school}[3]{
 \textbf{#1} #2 $\bullet$ #3\\
}
% END RESUME DEFINITIONS

\newcommand{\boldx}[1]{{\bf #1}}
%\newcommand{\boldx}[1]{{}}

%tikz
\usepackage{tikz}
\usetikzlibrary{shapes,arrows}

\newcommand{\gen}{\textbf{P1}\xspace}

\begin{document}

\small
\smallskip
\vspace*{-44pt}

\contact{\href{http://jasonlarkin.org}{\LARGE{Jason M Larkin, P}\lowercase{h}\LARGE{D}, Lead Quantum Computing Lab, Carnegie Mellon University SEI }}
{\href{mailto:jmlarkin@andrew.cmu.edu}{jmlarkin@andrew.cmu.edu} $\bullet$ active TS-SCI clearance}

%$\bullet$ 
%\textbf{Note: }\href{https://github.com/jasonlarkin/cmu-doc/raw/master/cv/stash/jasonlarkin_cv.pdf}
%{Red are hyperlinks in digital copy}}

\header{\large{Career Overview}}

I have 16 years experience in research and product development, working in the following domains:
\begin{itemize}
\item Quantum Computing and Information Science
\item High Performance Computing: DARPA, CMU-SEI QHub, \href{https://access-ci.org/}{ACCESS-CI} Network Research
\item Materials Science: atomistic$/$molecular modeling, nanoscale transport
\item Condensed Matter Physics}: turbulence$/$fluid dynamics %,nonlinear$/$visco-elasticity
%\item \textbf{Knowledge Management Systems}:natural language processing/understanding,"umwelt hacking"
%\item CyberPhysical:robotics$/$autonomy$/$assurance,"umwelt hacking"
\end{itemize}

\end{itemize}

\header{\large{Experience}}

\employer{\href{https://www.sei.cmu.edu/about/divisions/emerging-technology-center/}{CMU AI Division Software Engineering Institute}}{2017 - Present}{Senior Research Scientist}

%(2013-2015)\underline{\textbf{\emph{Software Engineer}}}

\vspace{1mm}
    \begin{itemize}
      \item \textbf{PROJECTS}
      \begin{itemize}
        \item {\href{https://www.sei.cmu.edu/our-work/quantum-computing/index.cfm}{\textbf{Quantum Advantage Evaluation Framework}}}
        \begin{itemize}
          \item PI or co-PI (5.8M funding) on applications for quantum advantage in combinatorial optimization, materials science, machine learning, cryptography. Products are implementations and tools for DOD stakeholders. 
          \item Technical and software engineering lead of a group tasked with modeling scalable fault-tolerant architectures and doing quantum resource estimation for transformative Applications. 
          \item Created \href{https://www.sei.cmu.edu/our-work/quantum-computing/index.cfm}{CMU-SEI QHub} supporting 15 researchers, publications, presentations, and courses at CMU. 
        \end{itemize}
        
        \item {\textbf{US DOD OUSD Quantum Advantage in NISQ Era} }}PI}
        \begin{itemize}
          \item Tasked to identify potential DOD Applications that can achieve Quantum Advantage in 1-3 years given projections of hardware progress. Delivered a guide to research for to DOD OUSD management over the next 1-3 and 5-10 year timeframe. 
          \item Formed long-term partnerships with DOD stakeholders (AFRL, NRL, ARL) with distinct Applications (combinatorial optimization, materials science, machine learning). Continue ongoing relationship in the form of shared research and funding partnerships. 
        \end{itemize}        


        \item {\textbf{{CMU NSF Quantum Computing Leap Initiative }} Co-PI}}
        \begin{itemize}
          \item Forming inter-discplinary consortium at CMU of relevant academic departments (e.g. materials science, computer science, computer engineering) with private sector partners (Lockheed Martin, Bosch, IBM, Google) to create a Quantum Computing Center focused on Application-level research. 
          \item 
        \end{itemize}        
        
        
        \item {\textbf{Pittsburgh Supercomputing Center NSF Novel Computing Platforms }}Co-PI}}
        \begin{itemize}
          \item Proposing a full-stack quantum computing simulation tool that will allow users to design scalable fault-tolerant quantum computers. 
          \item Working with private sector partners NVIDIA, Intel, and IBM to procure hardware.  I am working to help define the specifications of these classical compute resources given the various types of simulation needed to model fault-tolreant quantum computers.
          \item 
        \end{itemize}                
        
        \end{itemize}

        \item \textbf{\href{https://www.darpa.mil/program/software-defined-hardware}{Software Defined Hardware (SDH, \href{http://www.darpa.mil/default.aspx}{DARPA})(John Wohlbier PI)}}
        \begin{itemize}
            \item Created the testing infrastructure for SEI via AWS, which utilized compilation and analysis tools (Intel, PyTorch, Tensorflow, ARM) to establish maximum theoretical and empirical performance for data-intensive workflows (machine learning, graph analytics, optimization) on commodity hardware (CPU, GPU, TPU) to compare against new software-defined hardware designs. 
        \end{itemize}
   
        \item \textbf{GraphBLAS Test Framework (Scott McMillan PI)} 
        \begin{itemize}
            \item  Created a test framework for multiple implementations of GraphBLAS.org, including \href{https://github.com/cmu-sei/gbtl}{SEI’s GraphBLAS Template Library} https://github.com/cmu-sei/gbtl.
        \end{itemize}
        
                \item \textbf{\href{https://www.darpa.mil/program/software-defined-hardware}{Data Protection in Virtual Environments,\href{https://www.darpa.mil/program/data-protection-in-virtual-environments}{(DPRIVE, DARPA}) (Drew Dolgert PI)}}
        \begin{itemize}
            \item Project to utilize Fully Homomorphic Encryption (FHE) for practical Machine Learning workflows. I worked to create the tests/benchmarks for performers to demonstrate the goals of computational performance and security. These tests inform new architectural designs by the performers. 
        \end{itemize}
        
        
    \end{itemize}
    
    

\employer{\href{http://spiralgen.com/}{SpiralGen, Inc.}}{2013 - 2017}{Senior Research Engineer}

%(2013-2015)\underline{\textbf{\emph{Software Engineer}}}

\vspace{1mm}
%\par
%My work involved research and development across the of software and hardware that supported the Spiral toolchain for , with a focus on delivering  using (see \textbf{Skills} section for more):

%My focus of expertise is identified in \textbf{bold}.
%    \item Spanned range of languages from "nearly-assembly" C to JavaScript/Python/etc.  
    \begin{itemize}

      \item \textbf{PROJECTS}

      Supported work of commercial and research projects featuring the code-generation tool \textbf{\href{http://spiral.net/}{Spiral}}.

      \begin{itemize}

        \item \textbf{Spiral Code Generation Toolbox for Matlab/Simulink and Advanced Driver Assistance Systems (ADAS)}
        \begin{itemize} 
        \item Developed toolbox for Spiral code generation of Automotive Adaptive Cruise Control Using FMCW and MFSK Technology.
        \end{itemize}

        \item \textbf{\href{http://www.darpa.mil/Our_Work/I2O/Programs/High-Assurance_Cyber_Military_Systems_(HACMS).aspx}{High-Assurance Cyber Military Systems (HACMS, \href{http://www.darpa.mil/default.aspx}{DARPA})}} 
        \begin{itemize}
        Automatically-optimized / \href{https://en.wikipedia.org/wiki/Formal_methods}{formally-verified} kernels for \href{https://en.wikipedia.org/wiki/Cyber-physical_system}{Cyber-Physical Systems} using Spiral, plug-in for OSATE and the Architecture Analysis $\&$ Design Language (AADL), DARPA Demo Days ground/air vehicles, virtual/physical environments, Large/diverse collaboration team interacting with O(1000K) Lines of Code (LOC). 
        \end{itemize}
        
      \item \textbf{SpiralFFT for \href{http://www.ncsa.illinois.edu}{National Center for Supercomputing  (NCSA)} \href{http://www.ncsa.illinois.edu/enabling/bluewaters}{Blue Waters}} 
      \begin{itemize}
      \item Improved petascale performance of Hybrid \href{http://en.wikipedia.org/wiki/Message_Passing_Interface}{MPI} / \href{http://en.wikipedia.org/wiki/OpenMP}{OpenMP} \href{https://en.wikipedia.org/wiki/Fast_Fourier_transform}{FFT} and Stencils using Spiral. Applications include \href{https://en.wikipedia.org/wiki/Pseudo-spectral_method}{Pseudo Spectral Methods} for modeling turbulence and the NEURON simulation environment. 
      \end{itemize}

    \item \textbf{SpiralGen DevOps and Cloud Infrastructure}: 
    \begin{itemize}
    \item "Full-stack" software development environment for Spiral code generation. Provided Continuous Integration and Software Control Management for targeting many backends (e.g. Intel family, GPU, etc). Utilized early version of browser-based Integrated Development Environment (IDE) and virtualization on \textbf{\href{https://en.wikipedia.org/wiki/Amazon_Web_Services}{Amazon Web Services}} (AWS).  
    \item Provided integration with  \href{http://www.mathworks.com/products/matlab/}{Matlab}/\href{http://www.mathworks.com/products/simulink/}{Simulink}/\href{http://www.mathworks.com/help/matlab/ref/mex.html}{Mex}, \href{https://www.python.org/}{Python}/\href{http://cython.org/}{Cython}, \href{http://www.ros.org/}{ROS}, \href{https://www.cyberbotics.com/overview}{Webots}, \href{http://www.ls.cs.cmu.edu/KeYmaeraX/}{KeyMaeraX}, \href{http://www.wolfram.com/mathematica/}{Mathematica}.
    \end{itemize}


    \item \textbf{\href{http://www.darpa.mil/Our_Work/MTO/Programs/Power_Efficiency_Revolution_for_Embedded_Computing_Technologies_(PERFECT).aspx}{Power Efficiency Revolution for Embedded Computing Technologies (PERFECT, \href{http://www.darpa.mil/default.aspx}{DARPA})}}Eclipse RCP \textbf{first commerical release} of \href{http://www.spiralgen.com/new-products/spiral-fft-gpl-10}{SpiralFFT}.  

    \item \textbf{\href{http://www.darpa.mil/program/building-resource-adaptive-software-systems}{Building Resource Adaptive Sotware Systems} (BRASS, \href{http://www.darpa.mil/default.aspx}{DARPA})}:Test harness for Spiral-generated resource adaptive FFT for Synthetic Apeture Radar. 

  \end{itemize}

\item \textbf{FUNDING AND RESOURCE PROPOSALS}
        \begin{itemize}
          \item \textbf{DOD 172-008 SBIR} (co-wrote Phase 1).
          \item \textbf{\href{http://science.energy.gov/sbir/}{DOE SG-13808 SBIR}} (co-wrote, \href{https://www.osti.gov/scitech/servlets/purl/1242462}{Phase 1} awarded, Phase 2 submitted, denied).
          \item \textbf{DOD A15-102 SBIR} (PI, Phase 1 submitted).
          \item \textbf{\href{https://bluewaters.ncsa.illinois.edu/paid-ime-submissions}{NSF NCSA Blue Waters PAID IME Submission}} (Co-PI).
          \item \textbf{Optimization of 3-D FFTs for Intel Xeon Phi and NVIDIA Kepler K20 GPUs using Spiral} (PI, awarded).
        \end{itemize}


    \end{itemize}

%\end{achievements}


\employer{\href{https://www.cmu.edu/me/}{Carnegie Mellon University}}{2010-2012}{TA-Heat Transfer: lectured on conduction, convection, radiation.}
%	\begin{achievements}
%  \begin{itemize}
%    \item 
%  \end{itemize}
%	\end{achievements}

\employer{\href{http://www.engineering.pitt.edu/MEMS/}{University of Pittsburgh}}{2008}{TA-Fluid Mechanics: viscous, boundary, scale similarity, dimensional analysis.}
%  \begin{achievements}
%  \begin{itemize}
%    \item 
%  \end{itemize}
%  \end{achievements}

\employer{\href{http://www.physicsandastronomy.pitt.edu/}{University of Pittsburgh}}{2007-2009}{Lecturer-Physics: mathematics, turbulence, statistics and nonlinearity.}
%  \begin{achievements}
%  \begin{itemize}
%    \item Lectured on mathematics, bio-physics, turbulence, statistical and nonlinear phenomena. 
%  \end{itemize}
%  \end{achievements}

\employer{\href{http://www.precisiontherapeutics.com/}{Precision Therapeutics}}{2006-2007}{Intern-Technology Development: optical microscope automation design.} 
%  \begin{achievements}
%  \begin{itemize}
%    \item  
%  \end{itemize}
%  \end{achievements}

\header{\large{Education}}

\begin{itemize}[leftmargin=*]
  \item Carnegie Mellon University Pittsburgh, PA PhD Mechanical Engineering, 2013 GPA: 3.9/4.0}
%\schoolwithcourses{Carnegie Mellon University}{Pittsburgh, PA}{PhD Mechanical Engineering, 2013 GPA: 3.85/4.00}
%\area{
  \begin{itemize}
    \item Thesis: \href{http://jasonlarkin.github.io/projects-phd.html}
    {Vibrational Mode Properties of Disordered Solids from High-Performance Atomistic Simulations.}}
%Numerically investigated thermal properties of crystal alloys, glasses, and organic materials using classical and \href{http://en.wikipedia.org/wiki/Ab_initio_quantum_chemistry_methods}{\emph{ab initio}}-based atomistic techniques.
% Information Content: 1-10 GB
    \item  {\href{http://ntpl.me.cmu.edu/research.html}{Nanostructure Thermal Conductivity:} investigator for  
\href{http://www.wpafb.af.mil/afrl/afosr/}{AFOSR} on the \href{http://www.hpcmo.hpc.mil/cms2/index.php}{DOD's HPCMP}.}
    \item \href{https://nanochemistry.curtin.edu.au/local/docs/gulp/gulp4.2_manual.pdf}{GULP:} international collaboration with \href{http://nanochemistry.curtin.edu.au/people/staff.cfm/J.Gale}{Julian Gale} at the 
\href{http://nanochemistry.curtin.edu.au/}{Nanochemistry Research Institute} at \href{http://www.curtin.edu.au/}{Curtin University}.
  \end{itemize}

  \item \textbf{University of Pittsburgh} Pittsburgh, PA MS Mechanical Engineering, 2009 GPA: 3.7/4.0
  \begin{itemize}
    \item \textbf{Thesis}: \href{http://jasonlarkin.github.io/projects-ms.html}{Statistics of Particle Concentrations in Free-Surface Turbulence.} 
    \item \textbf{\href{http://jasonlarkin.github.io/projects-ms.html}{Statistics of Free-Surface Turbulence}:} \textbf{international} collaboration with \href{http://perso.ens-lyon.fr/alain.pumir/Pumir_webpage.html}{Alain Pumir} and \href{https://groups.oist.jp/ciu/mahesh-m-bandi}{Mahesh M. Bandi}. 
%Performed experiments using novel 2D and 3D flow configurations to study turbulence as a nonlinear dynamical system. 
  \end{itemize}

  \item \textbf{University of Pittsburgh} Pittsburgh, PA BS Mechanical Engineering, 2007 GPA: 3.2/4.0
  \begin{itemize}
    \item \textbf{Research}: \href{http://en.wikipedia.org/wiki/Finite_element_method}{FEM} design of model arterial bifurcation.
  \end{itemize}

  \item \textbf{Steel Center AVTS} Jefferson Hills, PA \href{http://en.wikipedia.org/wiki/Computer-aided_design}{CADD} Certification, 2002 GPA: 3.80/4.00

\end{itemize}

\header{\Large{Skills/Tools}}

%\begin{achievements}

\begin{itemize}[leftmargin=*]

  \item \textbf{Publication and Public Speaking}: \href{https://scholar.google.com/citations?hl=en&user=ry2klw0AAAAJ}{google scholar} (journal pubs (18), book chapters (2), conference presentations (28).

  \item \textbf{"Full-Stack" Software Engineering (stacks)}: 
  \begin{itemize}
    \item Qiskit: Aqua, Terra, Ignis, Pulse, Metal
    \item Microsoft Q#
    \item More limited experience with IntelQS, Cirq, Quimb, tkey, pyZX, STIM, XQSim
    \item Python/C++/C (PyTorch/Tensforflow/NLTK/scipy/numpy)
    \item Matlab-Simulink/C++/C/Fortran    
    \item \textbf{Software Configuration Management}: svn, git, GitHub, Jenkins, JIRA.  \textbf{Compilers/Compilation}: GNU, Intel, Visual Studio, Cython, Mex, make, cmake, MSBuild, Maven. 
    \item \textbf{Cloud Computing}: Amazon Web Service (AWS),  Azure, Docker, VirtualBox/VMWare, Ubuntu, Red Hat, CentOS, CoreOS, Windows (XP, 7, 8, Server). MPI / OpenMP, Vector Intrinsics (SSE/AVX/etc), CoArray Fortran
  \end{itemize}

  \item \textbf{Hardware:} optics/lasers, DI/DO AI/AO interfaces, automation, machining, circuitry,  robotics control.
 
\end{itemize}


\end{itemize}
%\end{achievements}

\header{\Large{\href{http://jasonlarkin.github.io/pub.html}{Publications} (selected, 27 total)}}
%\begin{achievements}
\begin{itemize}[leftmargin=*]
\item Evaluation of Quantum Approximation Optimization Algorithm, 
J Larkin, et al, Quantum Science and Technology (2022)
\item "Quantum Circuit Generation with SPIRAL", S. Mionis, F. Franchetti, J. Larkin, IEEE High Performance Extreme Computing Conference Septemeber 2021.
\item "Quantum Circuit Optimization with SPIRAL: A First Look", S. Mionis, F. Franchetti, J. Larkin, Supercomputing 2020.
\item "Assessment of Alternative Objective Functions for Quantum Variational Combinatorial Optimization ", M. Jonsson, J. Larkin, G. Guerreschi, IEEE QCE Quantum Week 2020.
\item Achieving the Quantum Advantage in Software, \href{https://insights.sei.cmu.edu/sei_blog/2019/12/achieving-the-quantum-advantage-in-software.html}{SEI Blog, (2019)}
\item "Projecting Quantum Advantage versus Classical State-of-the-Art", J. Larkin, D. Justice, IEEE HPEC 2019.

\item \href{http://www.cs.cmu.edu/~mmv/papers/17HACMS-Spiral.pdf}{High-Assurance SPIRAL: End-to-End Guarantees for Robot and Car Control, IEEE Control Systems, 2017}. 

\item Thermal Conductivity Accumulation in a-Si from Dynamic Simulation,
A. Benehban, JM Larkin, (in progress)
\item Reduced thermal conductivity of Si/Ge random layer nanowires,
N Samaraweera, JM Larkin, Journal of App. Physics (2018)
\item Thermal conductivity accumulation in amorphous silica and silicon, 
JM Larkin, et al, Physical Review B 89 (14), 144303 (2014)
\item Origins of thermal conductivity changes in strained crystals, 
KD Parrish, A Jain, JM Larkin, WA Saidi, AJH McGaughey Physical Review B 90 (23), 235201
\item Predicting alloy vibrational mode properties using lattice dynamics calculations, molecular dynamics simulations, and the virtual crystal approximation, JM Larkin, AJH McGaughey Journal of Applied Physics 114 (2), 023507
\item Disruption of superlattice phonons by interfacial mixing, SC Huberman, JM Larkin, AJH McGaughey, CH Amon Physical Review B 88 (15), 155311
%\item Decorrelating a Compressible Turbulent Flow: an Experiment, \href{http://pre.aps.org/abstract/PRE/v82/i1/e016301}{Physical Review E 82, 016301 (2010).}
\item Power-law distributions of particle concentration in free-surface flows, 
J Larkin, et al, Physical Review E 80 (6), 066301 (2009)
\end{itemize}
%\end{achievements}

\header{\Large{\href{http://jasonlarkin.org/pres.html}{Presentations} (selected, 28 total)}}
%\begin{achievements}
\begin{itemize}[leftmargin=*]
\item Effect of noise models on QAOA performance for Max-Cut, R. Majumdar, J. Larkin, G. Guerreschi, S. Susmita, APS March Meeting 2021.
\item Evaluation of QAOA, J. Larkin(speaker), \href{https://www.youtube.com/watch?v=I7yp-qXk0Zo}{Association Quantum}, \href{https://www.meetup.com/Washington-Quantum-Computing-Meetup/events/272829277/}{DC Quantum Meetup 2020}.
\item Quantum Circuit Optimization with SPIRAL, S. Mionis, J. Larkin, et al, Supercomputing 2020 (best presentation noimnee).
\item Assesing Objective Functions for Quantum Variational Optimization, M. Jonsson, J. Larkin et al, IEEE Quantum Week 2020.
\item Projecting NISQ-era quantum advantage with QAOA
GG Guerreschi, J Larkin, et al American Physical Society 65 2020.
\item \href{https://bluewaters.ncsa.illinois.edu/documents/10157/5a0a0d37-95bf-460b-a7f0-cfadd15abec8}{SpiralFFT for Blue Waters}, \textbf{J.M. Larkin} \href{https://www.youtube.com/watch?v=rjbEWeu2Nwc&feature=youtu.be#t=51m53s}{\textbf{(speaker)}}, \href{https://bluewaters.ncsa.illinois.edu/paid-ime#SPIRAL FFT}{NCSA Symposium for Petascale 2015}.
\item “Predicting Vibrational Mean Free Paths in Disordered Systems”, J.M. Larkin, A.J.H. McGuaghey,  presented at 2013 ASME Summer Heat Transfer Conference Minneapolis, MN.
\item “Effect of Interspecies Mixing on Phonon Mean Free Paths in Superlattices”, S.C. Huberman, J.M. Larkin, A.J.H. McGaughey, C.H. Amon, presented at 2013 ASME Summer Heat Transfer Conference Minneapolis, MN.
\item “Origin of Thermal Conductivity Changes in Strained Systems”, K. Parrish, J.M. Larkin, A.J.H. McGaughey, presented at 2013 ASME Summer Heat Transfer Conference Minneapolis, MN.
\item Virtual Crystal Approximation, \textbf{J.M. Larkin (speaker)}, \href{http://www.mrs.org/spring2013/}{2013 MRS Spring Meeting} San Francisco, CA.
\item “Ordered and Disordered Contributions to Lattice Thermal Conductivity”, J.M. Larkin (speaker), A.J.H. McGaughey, presented at 2012 PHONONS Conference Ann Arbor, MI.
\item “Predicting Phonon Properties of Silicon from First-Principles Calculations”, J.M. Larkin, A.J.H. McGaughey (speaker), W.A. Al-Saidi, presented at 2012 ASME Summer Heat Transfer Conference Puerto Rico, USA.
\item Generalized Fractal Dimensions...Turbulence, \textbf{J.M. Larkin (speaker)}, \href{http://meetings.aps.org/Meeting/MAR08/Content/1017}{2008 American Physical Society March Meeting}.
\item Flow Chamber to Explore the Development of Cerebral Aneurysms, J. Larkin, et al, 2007 Biomedical Engineering Society. 
%PHONONS 2012 http://www-personal.umich.edu/~pipe/Phonons_2012_Abstract_Book.pdf
\end{itemize}
%\end{achievements}


%\header{\Large{Honors}}
%%\begin{achievements}
%\begin{itemize}[leftmargin=*]
%\item \href{http://www.asmeconferences.org/HT2012/}{2012 ASME MHNMT International Summer Heat Transfer Conference} \textbf{Top 5 Technical Paper}
%\item \href{http://www.cmu.edu/me/news/archive/2011/bennett-conference.html}{2011 Bennett Conference \textbf{Best Presentation}}
%\item \href{http://www.ices.cmu.edu/newsitem.asp?NewsID=749}{2011 ICES \textbf{Northrop-Gruman Fellow}}
%\item 2007-2009 \textbf{NSF Graduate Student Research Grant} University of Pittsburgh Department of Physics.
%\end{itemize}
%\end{achievements}

%\header{Memberships}
%%\begin{achievements}
%\begin{itemize}[leftmargin=*]
%\item American Physical Society $\cdot$ American Society of Mechanical Engineers 
%$\cdot$ Materials Research Society $\cdot$ Society of Industrial and Applied Mathematics $\cdot$ DOD High Performance Computing Modernization Program $\cdot$ NCSA Blue Waters PAID IME
%\end{itemize}
%%\end{achievements}

\end{document}
